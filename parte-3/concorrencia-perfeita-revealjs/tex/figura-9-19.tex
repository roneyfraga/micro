\documentclass[border=10pt]{standalone}
\usepackage{tikz}
\usepackage[utf8]{inputenc}
\usepackage[T1]{fontenc}

\usetikzlibrary{arrows.meta}
\usetikzlibrary{decorations.pathreplacing} % NECESSÁRIO para brace

\begin{document}

\begin{tikzpicture}[scale=1.1]

% Eixos
\draw[->] (0,0) -- (8,0) node[right]{Quantidade};
\draw[->] (0,0) -- (0,6) node[above]{Preço};

% Curva de demanda
\draw[blue, thick] (0.5,5.5) -- (7.5,0.5) node[right] {$D$};

% Curva de oferta original
\draw[red, thick] (0.5,0.8) -- (7.2,5.5) node[right] {$S$};

% Ponto de equilíbrio onde S = D
\filldraw (3.85,2.85) circle (2pt);  
\filldraw (4.4,3.8) circle (2pt);

% Linhas tracejadas horizontais PV, P0 e PC
\draw[dashed] (0,3.8) -- (4.4,3.8);
\draw[dashed] (0,2.85) -- (3.85,2.85);
\draw[dashed] (0,2.2) -- (4.4,2.2);

% Rótulos no eixo Y
\node[left] at (0,3.8) {$P_v$};
\node[left] at (0,2.85) {$P_0$};
\node[left] at (0,2.2) {$P_c$};

% Linhas tracejadas verticais Q0 e Q1
\draw[dashed] (3.85,0) -- (3.85,2.85);
\draw[dashed] (4.4,0) -- (4.4,3.8);

% Rótulos no eixo X
\node[below] at (3.85,0) {$Q_0$};
\node[below] at (4.4,0) {$Q_1$};

% Conexão vertical entre Pv e Pc (benefício do subsídio)
\draw[decorate,decoration={brace,amplitude=8pt,mirror}]
(4.7,2.2) -- (4.7,3.8) node[midway,left=8pt] {$s$};

\end{tikzpicture}

\end{document}

