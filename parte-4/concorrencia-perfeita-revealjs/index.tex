% Options for packages loaded elsewhere
% Options for packages loaded elsewhere
\PassOptionsToPackage{unicode}{hyperref}
\PassOptionsToPackage{hyphens}{url}
\PassOptionsToPackage{dvipsnames,svgnames,x11names}{xcolor}
%
\documentclass[
  portuguese,
  letterpaper,
  DIV=11,
  numbers=noendperiod]{scrartcl}
\usepackage{xcolor}
\usepackage{amsmath,amssymb}
\setcounter{secnumdepth}{-\maxdimen} % remove section numbering
\usepackage{iftex}
\ifPDFTeX
  \usepackage[T1]{fontenc}
  \usepackage[utf8]{inputenc}
  \usepackage{textcomp} % provide euro and other symbols
\else % if luatex or xetex
  \usepackage{unicode-math} % this also loads fontspec
  \defaultfontfeatures{Scale=MatchLowercase}
  \defaultfontfeatures[\rmfamily]{Ligatures=TeX,Scale=1}
\fi
\usepackage{lmodern}
\ifPDFTeX\else
  % xetex/luatex font selection
\fi
% Use upquote if available, for straight quotes in verbatim environments
\IfFileExists{upquote.sty}{\usepackage{upquote}}{}
\IfFileExists{microtype.sty}{% use microtype if available
  \usepackage[]{microtype}
  \UseMicrotypeSet[protrusion]{basicmath} % disable protrusion for tt fonts
}{}
\makeatletter
\@ifundefined{KOMAClassName}{% if non-KOMA class
  \IfFileExists{parskip.sty}{%
    \usepackage{parskip}
  }{% else
    \setlength{\parindent}{0pt}
    \setlength{\parskip}{6pt plus 2pt minus 1pt}}
}{% if KOMA class
  \KOMAoptions{parskip=half}}
\makeatother
% Make \paragraph and \subparagraph free-standing
\makeatletter
\ifx\paragraph\undefined\else
  \let\oldparagraph\paragraph
  \renewcommand{\paragraph}{
    \@ifstar
      \xxxParagraphStar
      \xxxParagraphNoStar
  }
  \newcommand{\xxxParagraphStar}[1]{\oldparagraph*{#1}\mbox{}}
  \newcommand{\xxxParagraphNoStar}[1]{\oldparagraph{#1}\mbox{}}
\fi
\ifx\subparagraph\undefined\else
  \let\oldsubparagraph\subparagraph
  \renewcommand{\subparagraph}{
    \@ifstar
      \xxxSubParagraphStar
      \xxxSubParagraphNoStar
  }
  \newcommand{\xxxSubParagraphStar}[1]{\oldsubparagraph*{#1}\mbox{}}
  \newcommand{\xxxSubParagraphNoStar}[1]{\oldsubparagraph{#1}\mbox{}}
\fi
\makeatother


\usepackage{longtable,booktabs,array}
\usepackage{calc} % for calculating minipage widths
% Correct order of tables after \paragraph or \subparagraph
\usepackage{etoolbox}
\makeatletter
\patchcmd\longtable{\par}{\if@noskipsec\mbox{}\fi\par}{}{}
\makeatother
% Allow footnotes in longtable head/foot
\IfFileExists{footnotehyper.sty}{\usepackage{footnotehyper}}{\usepackage{footnote}}
\makesavenoteenv{longtable}
\usepackage{graphicx}
\makeatletter
\newsavebox\pandoc@box
\newcommand*\pandocbounded[1]{% scales image to fit in text height/width
  \sbox\pandoc@box{#1}%
  \Gscale@div\@tempa{\textheight}{\dimexpr\ht\pandoc@box+\dp\pandoc@box\relax}%
  \Gscale@div\@tempb{\linewidth}{\wd\pandoc@box}%
  \ifdim\@tempb\p@<\@tempa\p@\let\@tempa\@tempb\fi% select the smaller of both
  \ifdim\@tempa\p@<\p@\scalebox{\@tempa}{\usebox\pandoc@box}%
  \else\usebox{\pandoc@box}%
  \fi%
}
% Set default figure placement to htbp
\def\fps@figure{htbp}
\makeatother


% definitions for citeproc citations
\NewDocumentCommand\citeproctext{}{}
\NewDocumentCommand\citeproc{mm}{%
  \begingroup\def\citeproctext{#2}\cite{#1}\endgroup}
\makeatletter
 % allow citations to break across lines
 \let\@cite@ofmt\@firstofone
 % avoid brackets around text for \cite:
 \def\@biblabel#1{}
 \def\@cite#1#2{{#1\if@tempswa , #2\fi}}
\makeatother
\newlength{\cslhangindent}
\setlength{\cslhangindent}{1.5em}
\newlength{\csllabelwidth}
\setlength{\csllabelwidth}{3em}
\newenvironment{CSLReferences}[2] % #1 hanging-indent, #2 entry-spacing
 {\begin{list}{}{%
  \setlength{\itemindent}{0pt}
  \setlength{\leftmargin}{0pt}
  \setlength{\parsep}{0pt}
  % turn on hanging indent if param 1 is 1
  \ifodd #1
   \setlength{\leftmargin}{\cslhangindent}
   \setlength{\itemindent}{-1\cslhangindent}
  \fi
  % set entry spacing
  \setlength{\itemsep}{#2\baselineskip}}}
 {\end{list}}
\usepackage{calc}
\newcommand{\CSLBlock}[1]{\hfill\break\parbox[t]{\linewidth}{\strut\ignorespaces#1\strut}}
\newcommand{\CSLLeftMargin}[1]{\parbox[t]{\csllabelwidth}{\strut#1\strut}}
\newcommand{\CSLRightInline}[1]{\parbox[t]{\linewidth - \csllabelwidth}{\strut#1\strut}}
\newcommand{\CSLIndent}[1]{\hspace{\cslhangindent}#1}

\ifLuaTeX
\usepackage[bidi=basic]{babel}
\else
\usepackage[bidi=default]{babel}
\fi
% get rid of language-specific shorthands (see #6817):
\let\LanguageShortHands\languageshorthands
\def\languageshorthands#1{}


\setlength{\emergencystretch}{3em} % prevent overfull lines

\providecommand{\tightlist}{%
  \setlength{\itemsep}{0pt}\setlength{\parskip}{0pt}}



 


\KOMAoption{captions}{tableheading}
\makeatletter
\@ifpackageloaded{caption}{}{\usepackage{caption}}
\AtBeginDocument{%
\ifdefined\contentsname
  \renewcommand*\contentsname{Índice}
\else
  \newcommand\contentsname{Índice}
\fi
\ifdefined\listfigurename
  \renewcommand*\listfigurename{Lista de Figuras}
\else
  \newcommand\listfigurename{Lista de Figuras}
\fi
\ifdefined\listtablename
  \renewcommand*\listtablename{Lista de Tabelas}
\else
  \newcommand\listtablename{Lista de Tabelas}
\fi
\ifdefined\figurename
  \renewcommand*\figurename{Figura}
\else
  \newcommand\figurename{Figura}
\fi
\ifdefined\tablename
  \renewcommand*\tablename{Tabela}
\else
  \newcommand\tablename{Tabela}
\fi
}
\@ifpackageloaded{float}{}{\usepackage{float}}
\floatstyle{ruled}
\@ifundefined{c@chapter}{\newfloat{codelisting}{h}{lop}}{\newfloat{codelisting}{h}{lop}[chapter]}
\floatname{codelisting}{Listagem}
\newcommand*\listoflistings{\listof{codelisting}{Lista de Listagens}}
\makeatother
\makeatletter
\makeatother
\makeatletter
\@ifpackageloaded{caption}{}{\usepackage{caption}}
\@ifpackageloaded{subcaption}{}{\usepackage{subcaption}}
\makeatother
\usepackage{bookmark}
\IfFileExists{xurl.sty}{\usepackage{xurl}}{} % add URL line breaks if available
\urlstyle{same}
\hypersetup{
  pdftitle={Concorrência Perfeita},
  pdfauthor={Roney Fraga Souza},
  pdflang={pt},
  colorlinks=true,
  linkcolor={blue},
  filecolor={Maroon},
  citecolor={Blue},
  urlcolor={Blue},
  pdfcreator={LaTeX via pandoc}}


\title{Concorrência Perfeita}
\usepackage{etoolbox}
\makeatletter
\providecommand{\subtitle}[1]{% add subtitle to \maketitle
  \apptocmd{\@title}{\par {\large #1 \par}}{}{}
}
\makeatother
\subtitle{Gráficos do Pindyck e Rubinfeld (2013), Capítulo 9}
\author{Roney Fraga Souza}
\date{2025-11-19}
\begin{document}
\maketitle


\subsection{testes}\label{testes}

This display LaTeX formmula will be imagified:

\[\binom{n}{k} = \frac{n!}{k!(n-k)!}\]

As well as this TikZ picture:

\begin{tikzpicture}
\draw (-2,0) -- (2,0);
\filldraw [gray] (0,0) circle (2pt);
\draw (-2,-2) .. controls (0,0) .. (2,-2);
\draw (-2,2) .. controls (-1,0) and (1,0) .. (2,2);
\end{tikzpicture}

And this raw LaTeX block:

\subsection{load}\label{load}

\begin{figure}

\centering{

\emph{\textless LaTeX content not imagified\textgreater{}}
\textbackslash documentclass\{standalone\}
\textbackslash usepackage\{tikz\}
\textbackslash usepackage{[}utf8{]}\{inputenc\}
\textbackslash usepackage{[}T1{]}\{fontenc\}
\textbackslash usetikzlibrary\{patterns,arrows.meta\}

\textbackslash begin\{document\}

\textbackslash begin\{tikzpicture\}{[}scale=1.2, \textgreater=Stealth{]}
    
    \% Eixos
    \textbackslash draw{[}thick,-\textgreater{]} (0,0) -\/- (8,0) node{[}below{]} \{Quantidade\};
    \textbackslash draw{[}thick,-\textgreater{]} (0,0) -\/- (0,6.5) node{[}above{]} \{Preço\};
    
    \% Curvas de oferta e demanda
    \textbackslash draw{[}blue, very thick, domain=0:7{]} plot (\textbackslash x, \{6.5 - 0.9*\textbackslash x\}) node{[}right{]} \{\$D\$\};
    \textbackslash draw{[}red!70!black, very thick, domain=0:6.5{]} plot (\textbackslash x, \{1.5 + 0.6*\textbackslash x\}) node{[}right{]} \{\$S\$\};
    
    \% Ponto de equilíbrio
    \textbackslash coordinate (E) at (3.33, 3.5);
    
    \% Excedente do consumidor (área azul claro)
    \textbackslash fill{[}blue!20, opacity=0.7{]} (0,6.5) -\/- (0,3.5) -\/- (3.33,3.5) -\/- cycle;
    
    \% Excedente do produtor (área bege)
    \textbackslash fill{[}orange!30!yellow!30, opacity=0.8{]} (0,1.5) -\/- (0,3.5) -\/- (3.33,3.5) -\/- cycle;
    
    \% Linhas tracejadas horizontais (tocando as curvas)
    \textbackslash draw{[}dashed, gray{]} (0,5.5) node{[}left{]} \{US\textbackslash\$ 10\} -\/- (1.11,5.5);
    \textbackslash draw{[}dashed, gray{]} (1.11,5.5) -\/- (1.11,0);
    \textbackslash draw{[}dashed, gray{]} (0,4.1) node{[}left{]} \{US\textbackslash\$ 7\} -\/- (2.67,4.1);
    \textbackslash draw{[}dashed, gray{]} (2.67,4.1) -\/- (2.67,0);
    \textbackslash draw{[}dashed, gray{]} (0,3.5) node{[}left{]} \{US\textbackslash\$ 5\} -\/- (3.33,3.5);
    \textbackslash draw{[}dashed, gray{]} (3.33,3.5) -\/- (3.33,0);
    
    \% Marcações no eixo x
    \textbackslash draw (1.11,-0.1) -\/- (1.11,0.1) node{[}below, rotate=45, anchor=north east, yshift=-2mm{]} \{Consumidor \$A\$\};
    \textbackslash draw (2.67,-0.1) -\/- (2.67,0.1) node{[}below, rotate=45, anchor=north east, yshift=-2mm{]} \{Consumidor \$B\$\};
    \textbackslash draw (3.33,-0.1) -\/- (3.33,0.1) node{[}below, rotate=45, anchor=north east, yshift=-2mm{]} \{Consumidor \$C\$\};
    \textbackslash draw (3.33,0) node{[}below, yshift=-10mm{]} \{\$Q\_1\$\};
    
    \% Rótulos dos excedentes
    \textbackslash node{[}align=center{]} at (4.5, 5.5) \{\textbackslash small Excedente\textbackslash\textbackslash\textbackslash small do consumidor\};
    \textbackslash node{[}align=center{]} at (4.5, 1.5) \{\textbackslash small Excedente\textbackslash\textbackslash\textbackslash small do produtor\};
    
    \% Setas apontando para as áreas
    \textbackslash draw{[}-\textgreater, thick{]} (4.0, 5.2) -\/- (2.0, 4.5);
    \textbackslash draw{[}-\textgreater, thick{]} (4.0, 1.8) -\/- (2.0, 2.5);
    
\textbackslash end\{tikzpicture\}

\textbackslash end\{document\}
\emph{\textless end of LaTeX content\textgreater{}}

}

\caption{\label{fig-1}Figure: a TikZ image}

\end{figure}%

\subsection*{}\label{section}
\addcontentsline{toc}{subsection}{}

\begin{center}
\pandocbounded{\includegraphics[keepaspectratio]{figura-9-01.png}}
\end{center}

\subsection*{}\label{section-1}
\addcontentsline{toc}{subsection}{}

\begin{center}
\pandocbounded{\includegraphics[keepaspectratio]{figura-9-02.png}}
\end{center}

\subsection*{}\label{section-2}
\addcontentsline{toc}{subsection}{}

\begin{center}
\pandocbounded{\includegraphics[keepaspectratio]{figura-9-03.png}}
\end{center}

\subsection*{}\label{section-3}
\addcontentsline{toc}{subsection}{}

\begin{center}
\pandocbounded{\includegraphics[keepaspectratio]{figura-9-05.png}}
\end{center}

\subsection*{}\label{section-4}
\addcontentsline{toc}{subsection}{}

\begin{center}
\pandocbounded{\includegraphics[keepaspectratio]{figura-9-07.png}}
\end{center}

\subsection*{}\label{section-5}
\addcontentsline{toc}{subsection}{}

\begin{center}
\pandocbounded{\includegraphics[keepaspectratio]{figura-9-08.png}}
\end{center}

\subsection*{}\label{section-6}
\addcontentsline{toc}{subsection}{}

\begin{center}
\pandocbounded{\includegraphics[keepaspectratio]{figura-9-10.png}}
\end{center}

\subsection*{}\label{section-7}
\addcontentsline{toc}{subsection}{}

\begin{center}
\pandocbounded{\includegraphics[keepaspectratio]{figura-9-11.png}}
\end{center}

\subsection*{}\label{section-8}
\addcontentsline{toc}{subsection}{}

\begin{center}
\pandocbounded{\includegraphics[keepaspectratio]{figura-9-14.png}}
\end{center}

\subsection*{}\label{section-9}
\addcontentsline{toc}{subsection}{}

\begin{center}
\pandocbounded{\includegraphics[keepaspectratio]{figura-9-15.png}}
\end{center}

\subsection*{}\label{section-10}
\addcontentsline{toc}{subsection}{}

\begin{center}
\pandocbounded{\includegraphics[keepaspectratio]{figura-9-17.png}}
\end{center}

\subsection*{}\label{section-11}
\addcontentsline{toc}{subsection}{}

\begin{center}
\pandocbounded{\includegraphics[keepaspectratio]{figura-9-18.png}}
\end{center}

\subsection*{}\label{section-12}
\addcontentsline{toc}{subsection}{}

\begin{center}
\pandocbounded{\includegraphics[keepaspectratio]{figura-9-19.png}}
\end{center}

\subsection{Referências}\label{referuxeancias}

\phantomsection\label{refs}
\begin{CSLReferences}{0}{0}
\bibitem[\citeproctext]{ref-pindyck2013}
PINDYCK, R. S.; RUBINFELD, D. L. \textbf{Microeconomia}. {[}s.l.{]}
Pearson Education do Brasil, 2013.

\end{CSLReferences}




\end{document}
